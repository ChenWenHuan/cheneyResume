\documentclass[11pt,a4paper]{moderncv}

% moderncv themes
%\moderncvtheme[blue]{casual}                 % optional argument are 'blue' (default), 'orange', 'red', 'green', 'grey' and 'roman' (for roman fonts, instead of sans serif fonts)
\moderncvtheme[blue]{classic}                % idem
\usepackage{xunicode, xltxtra}
\XeTeXlinebreaklocale "zh"
\widowpenalty=10000

%\setmainfont[Mapping=tex-text]{文泉驿正黑}

% character encoding
%\usepackage[utf8]{inputenc}                   % replace by the encoding you are using
\usepackage{CJKutf8}

% adjust the page margins
\usepackage[scale=0.8]{geometry}
\recomputelengths                             % required when changes are made to page layout lengths
\setmainfont[Mapping=tex-text]{Hiragino Sans GB}
\setsansfont[Mapping=tex-text]{Hiragino Sans GB}
\CJKtilde

% personal data

%% start of file `template-zh.tex'.
%% Copyright 2006-2012 Xavier Danaux (xdanaux@gmail.com).
%
% This work may be distributed and/or modified under the
% conditions of the LaTeX Project Public License version 1.3c,
% available at http://www.latex-project.org/lppl/.

% 个人信息
\firstname{陈}
\familyname{文焕}
%\title{Resume}                      % 可选项、如不需要可删除本行
\address{江宁将军大道10号}{211100 南京}         % 可选项、如不需要可删除本行
\mobile{+86~17768118624}                         % 可选项、如不需要可删除本行
%\phone{+2~(345)~678~901}                          % 可选项、如不需要可删除本行
%\fax{+3~(456)~789~012}                            % 可选项、如不需要可删除本行
\email{wenhuan\_chen@163.com} % 可选项、如不需要可删除本行
%\homepage{github.com/ChenWenHuan} % 可选项、如不需要可删除本行
%\address{ShengTai Road 28, Nanjing }{Blog: http://blog.csdn.net/cheneywh} % 可选项、如不需要可删除本行
%\fax{i.youku.com/juda1024} % 可选项、如不需要可删除本行
%\extrainfo{附加信息 (可选项)}                  % 可选项、如不需要可删除本行
\photo[64pt]{chenwenhuan.jpg}                  % ‘64pt’是图片必须压缩至的高度、‘0.4pt‘是图片边框的宽度 (如不需要可调节至0pt)、’picture‘ 是图片文件的名字;可选项、如不需要可删除本行
%\quote{引言(可选项)}                           % 可选项、如不需要可删除本行

% 显示索引号;仅用于在简历中使用了引言
%\makeatletter
%\renewcommand*{\bibliographyitemlabel}{\@biblabel{\arabic{enumiv}}}
%\makeatother

% 分类索引
%\usepackage{multibib}
%\newcites{book,misc}{{Books},{Others}}
%----------------------------------------------------------------------------------
%            内容
%----------------------------------------------------------------------------------
\begin{document}
\maketitle
\section{Introduction}
\cventry{}{专注嵌入式平台,移动设备软件开发十年}{}{}{}{}

\section{专业技能}
\cventry{1.}{语言}{精通C语言,熟悉C++,Java语言,了解汇编语言}{}{}{}{}
\cventry{2.}{应用开发}{Android,iOS的应用开发,架构。RxAndroid,RxBinding,Retrofit,etc}{}{}{}{}
\cventry{3.}{Android}{掌握Android系统原理和开发调试,熟悉Android的系统架构}{}{}{}{}
\cventry{4.}{Android APP}{掌握Android的App开发,并熟练运用常见的架构如:Jni架构,C/S架构}{}{}{}
\cventry{5.}{Linux}{熟悉Linux应用开发,linux驱动(BSP)的开发调试工作。熟悉和掌握常见的调试和分析工具:makefile,gdb,objdump,elf等}{}{}{}
\cventry{6.}{脚本}{熟悉Linux Shell 和Python}{}{}{}{}
\cventry{7.}{算法}{良好的算法基础,了解一般的机器学习算法}{}{}{}{}
\cventry{8.}{版本管理}{精通Git版本管理工具,熟练使用SVN,ClearCase}{}{}{}
\cventry{9.}{代码编辑}{熟练使用Vim,Emacs,LATEX}{}{}{}{}


\section{社区}
\cventry{Blog}{\url{http://blog.csdn.net/cheneywh}}{}{}{}{}
\cventry{GitHub}{\url{https://github.com/ChenWenHuan}}{}{}{}{}
%\cventry{StackOverflow}{\url{http://stackoverflow.com/users/1983467/dinever}}{}{}{}{}

\section{项目经历}
\renewcommand{\baselinestretch}{1.2}

\cventry{2015.05至今}
{我的E家产品}
{Android app,iOS app}
{南京咱家网络科技}{}
{南京咱家网络成立于2015年5月,专注于家庭设备的互联,管理平台。\cvline{}{}{1.参与并且架构了android,iOS APP。} }



\cventry{2011至2015.5}
{HTC Engineer Tool}
{Android app,Cts,Native App}
{诚迈科技}{}
{Engineer tool 是一个专门用来为工程模式开发的一个android app,通过隐藏的按键模式进入这个app可以查看关于移动设备的信号强度等一系列工程信息。
\cvline{}{}{1.软件开发及维护}}

\cventry{}
{OMADM信息采集软件}
{Android \, COS}
{}{}{本软件可运行在Android和Cos(Cos是国产手机操作系统)的一个远程采集和更新用户信息的APP,主要应用了业界移动采集信息的标准协议omadm协议。通过本软件,可以实现远程开解锁,用户信息采集,软件安装及更新等作用 \cvline{}{}{1.协议中各项功能的实现}}


\cventry{}
{COS系统Framework开发}
{Android Framework,Cos}
{}{}
{中国自主操作系统\href{http://www.china-cos.com/site/index.html}{COS系统}的Framework开发,主要工作是参考了Android的Framework,利用C++重新实现了Android的Framework层,做到了与Android的接口兼容。\cvline{}{}{1.主要负责wifi Framework接口的开发与调试,并对android的升级进行跟踪,实现改写的同步升级。} \cvline{}{}{2.蓝牙部分模块如MediaScan,HID模块的改写,API调试,测试工作。}}

\cventry{}
{智能电视盒子产品开发}
{Android,全志平台}
{}{}{一个基于全志平台的智能电视棒产品,主要让就有USB接口的非智能电视拥有智能电视的功能,产品参考苹果TV,小米盒子等相关的产品.\cvline{}{}{1.主要负责对各种开源库的整合}\cvline{}{}{2.对项目代码进行管理,版本发布。} }

\cventry{}
{Ashare屏幕投影项目}
{Android Framework,Native App,Jni,协议}
{}{}
{Ashare是一个让不同平台,不同设备屏幕相互分享的项目。 \cvline{}{}{1.主要负责UPNP协议的整合} \cvline{}{}{2.负责libjpeg-turbo开源库的整合.} \cvline{}{}{3.利用SIMD进行编解码的加速工作。} \cvline{}{}{4.负责虚拟驱动的开发。} \cvline{}{}{5.负责本项目的专利文档撰写与申请。} }


\cventry{2010—2011}
{智能彩票择号引擎}{}{和朋友开了个工作室}{}{}
\cventry{}
{高速金融软件客户端的开发}{}{}{}{}
\cventry{}
{ 移动电子书阅读软件的开发}{}{}{}{}


\cventry{2007.11—2010.01}
{Linux Java平台CDMA手机的开发}{Linux,Java,Brew}{南京摩托罗拉}{}{LJ平台的智能手机。主要根据运营商需求进行功能定制。
\cvline{}{}{1.根据运营商提出的需求做出设计分析}\cvline{}{}{2.根据设计文档进行编码}\cvline{}{}{3. Bug Fixed}}

\cventry{}
{开发维护BREW平台下的JAVA平台}{Linux,Java,Brew}{}{}{开发维护BREW平台下的JAVA平台。
\cvline{}{}{1.测试Java平台在Brew平台中的稳定性,并提出需求给第三方}\cvline{}{}{2.根据Java平台的需求封装提供Brew系统的API}\cvline{}{}{3. Bug Fixed}}

\cventry{}
{高通BREW平台基础功能定制开发}{Brew}{}{}{根据运营商的需求进行电话本功能,Calling的定制。
\cvline{}{}{1.Fixed Bugs。}}

\cventry{2006.10—2007.10}
{高速嵌入式网络接入芯片平台开发}{Linux,keil C,asm}{上海福华微电子}{}{
\cvline{}{}{1.嵌入式网络芯片平台驱动整合及TCP/IP协议栈的测试。}\cvline{}{}{2.嵌入式网络应用程序如网络摄像头,网络收音机及协议转换RS232转TCP/IP的开发}}


\cventry{2005—2006.10}
{型PLC软件平台开发}{Linux,keil C,asm}{无锡光洋电子}{}{
\cvline{}{}{1.负责PLC相关软件的研发和维护。}\cvline{}{}{2.参与新型PLC的软件开发}\cvline{}{}{3.负责PLC语言转化的动态链接库编写}}


\section{语言技能}
\cvline{英语}{\textbf{CET-6},擅长读写}
\cvline{普通话}{母语}

\section{兴趣爱好}
\cvline{跑步}{\textbf{已完成两次半程马拉松}}
\cvline{打篮球}{得分后卫}
\cvline{看小说}{科幻,玄幻迷}

%\section{计算机技能}
%\cvline{编程语言}{Python == C > PHP == Javascript == Java  > C++ }
%\cvline{数据库}{MySQL, MongoDB}
%\cvline{Web框架}{Django, Tornado, Express(Node.js)}
%\cvline{工具}{Linux, Vim, Git, \LaTeX}


\section{教育背景}
\cventry{2001 -- 2005}{本科}{长春理工大学}{信息对抗专业}{}{}  % 第3到第6编码可留白

\closesection{}                   % needed to renewcommands
\renewcommand{\listitemsymbol}{-} % change the symbol for lists

% 来自BibTeX文件但不使用multibib包的出版物
%\renewcommand*{\bibliographyitemlabel}{\@biblabel{\arabic{enumiv}}}% BibTeX的数字标签
\nocite{*}
\bibliographystyle{plain}
\bibliography{publications}                    % 'publications' 是BibTeX文件的文件名

% 来自BibTeX文件并使用multibib包的出版物
%\section{出版物}
%\nocitebook{book1,book2}
%\bibliographystylebook{plain}
%\bibliographybook{publications}               % 'publications' 是BibTeX文件的文件名
%\nocitemisc{misc1,misc2,misc3}
%\bibliographystylemisc{plain}
%\bibliographymisc{publications}               % 'publications' 是BibTeX文件的文件名

\end{document}


%% 文件结尾 `template-zh.tex'.
